\documentclass{article}
%\input{../../../LatexHeader/headerKlausurDeutsch}
\begin{document}


Fakultät für Mathematik, Universität Duisburg-Essen Campus Duisburg\\

\begin{center}
\huge{
Klausur ReplaceSemester: Diskrete Mathematik für Informatiker 1} \\[2em]                                       \end{center}
\large

\begin{itemize}
\item Die Bearbeitungszeit dieser Klausur betr\"agt 120 Minuten.
\item Die Klausur besteht aus 5 Aufgaben. Begründen Sie Ihre Lösungen.
\item Legen Sie bitte Ihren Studierendenausweis zur Identit\"atskontrolle gut sichtbar
auf Ihren Arbeitsplatz.
\item  Als Hilfsmittel ist ein nicht programmierbarer Taschenrechner zugelassen sowie ein beidseitig \textbf{von Hand} beschriebenes DIN A4 Blatt.
\item  Zum Bestehen der Klausur sind 50 Punkte hinreichend.\\[1em]
\end{itemize}

\vspace{3em}

{\bf Aufgabe 1 (20 Punkte)} 
ReplaceProblem1

\vspace{2em}


{\bf Aufgabe 2 (20 Punkte)} 
ReplaceProblem2


\vspace{2em}
  
{\bf Aufgabe 3 (20 Punkte)} 
ReplaceProblem3

\vspace{2em}

{\bf Aufgabe 4 (20 Punkte)} 
ReplaceProblem4

 

\vspace{2em}

{\bf Aufgabe 5 (20 Punkte)} 
ReplaceProblem5


 
\end{document}
